\documentclass[10pt]{article}
\usepackage[utf8]{inputenc}
\usepackage{fullpage}
\usepackage{exsheets}

\linespread{1.1}
\setlength\parindent{0pt}

\begin{document}

The nonlinear pendulum.

The equation of motion for the pendulum using Newton's second law can be written in terms of the displacement angle $\theta$ as:
%
\begin{equation}\label{eq:pendulum}
\frac{d^2\theta}{dt^2} = - \frac{g}{l} \sin \theta
\end{equation}

We first use the trick to turn the second-order equation into two first-order equations by introducing a new variable $\omega$:
%
\begin{eqnarray}
\frac{d\theta}{dt} & = & \omega \label{eq:pendulum1}\\ 
\frac{d\omega}{dt} & = & -\frac{g}{l} \sin \theta \label{eq:pendulum2}
\end{eqnarray}

\begin{question}
Write a program to solve the equations~(\ref{eq:pendulum1}) and (\ref{eq:pendulum2}) using the second-order Runge-Kutta method for a pendulum with a 10 cm arm.
Calculate the angle $\theta$ of displacement for several periods of the pendulum starting from the initial condition $(\theta, \omega) = (179^\circ, 0)$.
\end{question}

\begin{question}
Make a plot of the total energy of the system as a function of time:
%
\begin{equation}
E = K + U = \frac{1}{2} m (\omega^2 L^2) - m g L (1 - \cos \theta) 
\end{equation}
\end{question}

\end{document}
