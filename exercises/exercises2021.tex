\documentclass[10pt]{article}
\usepackage[utf8]{inputenc}
\usepackage{fullpage}
\usepackage{exsheets}

\linespread{1.1}
\setlength\parindent{0pt}

\begin{document}

\section{Problem: The nonlinear pendulum}

The equation of motion for the pendulum using Newton's second law can be written in terms of the displacement angle $\theta$ as:
%
\begin{equation}\label{eq:pendulum}
\frac{d^2\theta}{dt^2} = - \frac{g}{l} \sin \theta
\end{equation}

We first use the trick to turn the second-order equation into two first-order equations by introducing a new variable $\omega$:
%
\begin{eqnarray}
\frac{d\theta}{dt} & = & \omega \label{eq:pendulum1}\\ 
\frac{d\omega}{dt} & = & -\frac{g}{l} \sin \theta \label{eq:pendulum2}
\end{eqnarray}

\begin{question}
Write a program to solve the equations~(\ref{eq:pendulum1}) and (\ref{eq:pendulum2}) using the second-order Runge-Kutta method for a pendulum with a 10 cm arm.
Calculate the angle $\theta$ of displacement for several periods of the pendulum starting from the initial condition $(\theta, \omega) = (179^\circ, 0)$.
\end{question}

\begin{question}
Make a plot of the total energy of the system as a function of time:
%
\begin{equation}
E = K + U = \frac{1}{2} m (\omega^2 L^2) + m g L (1 - \cos \theta) 
\end{equation}
\end{question}

\newpage

\section{Problem: The 1-D diffusion equation}

The 1-dimensional diffusion equation assuming constant diffusion coefficient and steady state is
%
\begin{equation}\label{eq:diffusion}
-D\frac{\partial^2 f(x)}{\partial x^2} = Q(x)
\end{equation}
%
where $x$ is a spatial coordinate between $[-H,H]$ and we assume as boundary conditions: $f(x = \pm 1) = 0$.

To provide us with an analytical solution (with the correct boundary conditions) we prescribe 
%
\begin{equation}
\tilde f(x) = \cos(\pi x / 2)
\end{equation}
%
to be an analytical solution and we derive the correspondent source term using equation~(\ref{eq:diffusion}):
%
\begin{equation}
Q(x) = \dots 
\end{equation}

Equipped now with a source term, we find numerically the solution of the equation~(\ref{eq:diffusion}) as a limit for $t \rightarrow \infty$ of the time-dependent equation:
%
\begin{equation}
\frac{\partial f(x,t)}{\partial t} -D \frac{\partial^2 f(x,t)}{\partial x^2} = Q(x)
\end{equation}

We do this by using the Crank-Nicolson method and by plotting as a function of time a measure (e.g., 2-norm) of the distance between the numerical solution and the analytical one.

\begin{question}
Using the parameter values $H = 1$, $D = 0.1$, $dt = 10^{-3}$, $N = 2^8$ the plot should look like the following fig.
\end{question}

\begin{center}
\includegraphics[scale=0.35]{DiffusionTimeEvolution.pdf}
\end{center}

\begin{question}
Show also the case for $N = 2^9$ and $N = 2^{10}$. What is the error ratio in the limit $t \rightarrow \infty$ between these cases?
\end{question}

\end{document}
